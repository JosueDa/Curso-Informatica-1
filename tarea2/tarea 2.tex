\documentclass[10pt,a4paper]{article}
\usepackage[utf8]{inputenc}
\usepackage[spanish]{babel}
\usepackage{amsmath}
\usepackage{amsfonts}
\usepackage{amssymb}
\author{Josué Morales 20181101}
\title{tarea 2 Informática 1}
\date{2 de Agosto del 2018}
\begin{document}
 \title{\bf Hoja de trabajo 2}
 \maketitle
{\bf Ejercicio 1:}
Demostrar usando inducción
  \begin{itemize}
  \item Caso base 
  \[ 0^3 \geq 0^2\]
  \[0 \geq 0\]  
 \item Ahora se demostrará con su sucesor
  \[(n+1)^3 \geq (n+1)^2\]
\[(n+1)(n+1)^2 \geq (n+1)^2\]
\[(n+1)\geq (n+1)^2/(n+1)^2\]  
\[n+1\geq 1\]  
\[n\geq 1-1\]  
\[n\geq 0\]  
  \end{itemize}

{\bf Ejercicio 2:} desigualdad de Bernoulli demostrada por inducción.

Donde $n\in \mathbb{N}$, $x\in \mathbb{Q}$ y $x\geq -1$
\begin{itemize} 
      	\item caso base donde n= 0 y x= 0
\[(1+x)^n\geq nx\]
\[(1+0)^0\geq 0\]   
\[1^0\geq 0\]   
\[1\geq 0\]  
\item \[(1+x)^n\geq nx + 1 \] 
\item Multiplicar por (1+n) cada lado de la desigualdad
\item \[1+n\geq nx^2\]
  	  \end{itemize}
  	
  	

 \end{document}