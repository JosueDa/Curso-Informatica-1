\documentclass[10pt,a4paper]{article}
\usepackage[utf8]{inputenc}
\usepackage[spanish]{babel}
\usepackage{amsmath}
\usepackage{amsfonts}
\usepackage{amssymb}
\author{Josué Morales 20181101}
\title{tarea 1 Informática 1}
\date{26 de Junio del 2018}
\begin{document}
 \title{\bf Hoja de trabajo 1}
 \maketitle
{\bf Ejercicio 2:}
 
  \begin{itemize}
   \item{Conjunto de nodos: $\{1,2,3,4,5,6\}$}
   
   \item Conjunto de vértices del grafo
  $$
        \left\langle \left\{
            \begin{bmatrix}
                \langle 1,2 \rangle & \langle 1, 3\rangle & \langle 1,4   \rangle \\
                \langle 1,5 \rangle & \langle 2,3 \rangle & \langle 2,4 \rangle \\
                \langle 2,6 \rangle & \langle 3,5 \rangle & \langle 3,6 \rangle \\
                \langle 4,5 \rangle & \langle 4,6 \rangle & \langle 5,6 \rangle \\
            \end{bmatrix}
        \right\} \right\rangle
    $$ \\
  \end{itemize}

  {\bf Ejercicio 3:}
    \begin{itemize}
  	\item La estructura que puede representar un lanzamiento de dados es una estructura de camino.
  	
  	\item El algoritmo sería uno que diera números al azar, del dado, pero no pueda dar de nuevo el dato en que se encuentra ubicado, solamente las caras que le quedarán posibles.
  	
  	\item Siempre que el camino sea al azar y tenga un final.
  	
  	\end{itemize}
  	
  	

 \end{document}