\documentclass[10pt,a4paper]{article}
\usepackage[utf8]{inputenc}
\usepackage[spanish]{babel}
\usepackage{amsmath}
\usepackage{amsfonts}
\usepackage{amssymb}
\author{Josué Morales 20181101}
\title{tarea 1 Informática 1}
\date{30 de Agosto del 2018}
\begin{document}
 \title{\bf Hoja de trabajo 4}
 \maketitle
{\bf Ejercicio 1:}
Indicar que definiciones pertenecen al mismo conjunto:

\begin{enumerate}
	\item{$a:=\{1,2,4,8,16,32,64\}$}
	\item{$b:=\{n\ \in \mathbb{N}\ |\ \exists x \in \mathbb{N}\ .\ x=n/5 \}$}
	\item{$c:=\{n\in \mathbb{N}\ |\ \exists x\in\mathbb{N}\ .\ n=x*x \}$}
	\item{$d:=\{n\in\mathbb{N}\ |\ \exists i\in\mathbb{N}\ .\ n=2^i\wedge n<100 \}$}
	\item{$e:=\{ n\in\mathbb{N}\ |\ \exists x\in \mathbb{N}\ .\ x=\sqrt{n} \}$}
	\item{$f:=\{ n\in\mathbb{N}\ |\ \exists x\in \mathbb{N}\ .\ n=x+x+x+x+x \}$}
\end{enumerate}

{\bf Respuestas:}
\begin{itemize}
	\item D corresponde a A
	\item E corresponde a C
	\item B corresponde a F 
\end{itemize}
  
 \
 
{\bf Ejercicio 2:} definir los siguientes conjuntos con jerga matemática
\
\begin{enumerate}
	\item{El conjunto de todos los naturales divisibles dentro de $5$}
	\item{El conjunto de todos los naturales divisibles dentro de $4$ y $5$}
	\item{El conjunto de todos los naturales que son primos}
	\item{El conjunto de todos los conjuntos de numeros naturales que contienen
		un numero divisible dentro de $15$}
	\item{El conjunto de todos los conjuntos de numeros naturales que al ser sumados
		producen $42$ como resultado}
\end{enumerate}
\
{\bf Respuestas:}
\begin{enumerate} 
	
	\item{$A:=\{n\ \in \mathbb{N}\ |\ \exists x \in \mathbb{N}\ .\ x=n/5 \}$}
	
	\item{$A \cap B:=\{x/x\ \in A\ \wedge  \in B \}$} 
	
	{$A:=\{n\ \in \mathbb{N}\ |\ \exists x \in \mathbb{N}\ .\ x=n/5 \}$}
	
	{$B:=\{n\ \in \mathbb{N}\ |\ \exists x \in \mathbb{N}\ .\ x=n/4 \}$}
	
	\item {$C:= \{ \forall \ n \in \mathbb{N}\ | \ \nexists  x \in \mathbb{N}\ \ 1 < x < n \ . \ n \ mod(x) = 0 \} $}
	
	\item 	{$D:=\{n\ \in \mathbb{N}\ |\ \exists x \in \mathbb{N}\ .\ n=x*15 \}$}
	
	\item 	{$E:=\{n\ \in \mathbb{N}\ | \sum_{i=1}^{|42|} n_i = 42 \}$}
	
 \end{enumerate}
   \
  
  {\bf Ejercicio 3:} 
  \\
  Definir una relaci\'on llamada $S\subset \mathbb{N}_{50}\times\mathbb{N}_{50}\times\mathbb{N}_{50}$ en donde $\mathbb{N}_{30}:=\{ n \in \mathbb{N}\ |\ n\leq 30 \}$. La cual relaciona a todos los numeros
  \emph{semi-primos} menores a $30$ con los numeros primos que lo forman. Las tripletas que pertencen
  al conjunto que define dicha relaci\'on deben ser de la forma $\langle \mathtt{primo}_1,\mathtt{primo}_2,
  \mathtt{semi-primo} \rangle$, por ejemplo, para el numero $6$ corresponderia la tripleta $\langle 2,3,6 \rangle$
  
  	
  	\
  	
  	 {\bf Respuesta:}
  	 
  	 \begin{itemize}
  	   	
   	\item Definición por extensión 
   	
   	$$
   	{N}_{30}:= \left\{
   	\begin{bmatrix}
   	\langle 2,2,4 \rangle , & \langle 2,3,6 \rangle, & \langle 3,3,9 \rangle, \\
   	\langle 2,5,10 \rangle, & \langle 2,7,14 \rangle, & \langle 3,5,15 \rangle, \\
   	\langle 3,7,21 \rangle, & \langle 2,11,22 \rangle, & \langle 5,5,25 \rangle, \\
   	\langle 2,3,26 \rangle
   	\end{bmatrix}
   	\right\}
   	$$ \\
   	
  
  	 \end{itemize} 
   
    {\bf Ejercicio 4:} 
  \
  Definir los conjuntos de las siguientes funciones:
  
  \begin{enumerate}
  	\item{$f:\mathbb{N}\rightarrow\mathbb{N}$; $f(x)=x+x$}
  	\item{$g:\mathbb{N}\rightarrow\mathbb{B}$; $g(x)$ es verdadero si
  		$x$ es divisible dentro de $5$, falso en caso contrario. Nota: $\mathbb{B}=
  		\{\mathtt{true},\mathtt{false}\}$, puede definir dos conjuntos separados y
  		definir la funci\'on como la union de ambos conjuntos.}
  	\item{Indicar el conjunto al que pertenece la funci\'on $f\circ g$}
  	\item{Definir el conjunto que corresponde a la funci\'on $f\circ g$}
  \end{enumerate}
  
      {\bf Respuestas:}
       
  \begin{enumerate}
  	
  	\item  $ \ f \  = \{ x \in \ \mathtt{N} \ |\  (x , x+x) \} $ 
  	
  	\item {$a \cup b:=\{ \lambda (true,false)\ \in a\ \wedge  \in b \}$} 
  	\\
  	\\ {$a:=\{(n,true)\ | n \in \mathbb{N} \ \wedge\ \exists x \in \mathbb{N}\ .\ x=n/5 \}$}
  	\\ {$b:=\{(n,false)\ | n \in \mathbb{N} \ \wedge\ \exists x \in \mathbb{N}\ .\ x= \neg (n/5) \}$}
  	
  	\item {$f\circ g \in \ C=( n \in \ \mathtt{N} \ | 2n ) $}
  	
  	\item 	{$D:=\{(n,f(g)) \ | n \in \mathbb{N} \wedge f(x)\in\mathbb{N} \wedge g(x) \subset f(x) \}$}
  	
  	
  \end{enumerate}	

{\bf Ejercicio 5:}
\ 
dadas las siguientes funciones que pertenecen a $\mathbb{R}\rightarrow \mathbb{R}$, indique si la funci\'on es injectiva, surjectiva o bijectiva.

\begin{enumerate}
	\item{$f(x)=x^2$}   {\bf Surjectiva} 
	\item{$g(x)=\frac{1}{cos(x-1)}$}  {\bf Inyectiva} 
	\item{$h(x)=2x$}  {\bf Biyectiva}
	\item{$w(x)=x+1$}  {\bf Biyectiva}
\end{enumerate}

\textbf{Ejercicio 6}

\begin{itemize}
	\item {$B1:=\{(x,y) \ | (x,y) \in \mathbb{N} > 0 \ \wedge \exists n \in \mathbb{N}. x=2n \}$}
	
	\item {$B2:=\{(x,y) \ | (x,y) \in \mathbb{N} > 0 \ \wedge \exists n \in \mathbb{N}. x=(2n-1) \}$}
	
	\item {$C:=\{(x,y) \ | (x,y) \in \mathbb{Z}^-\ \wedge \exists n \in \mathbb{Z}^-. x= (2n-1) \}$}
	
	\item {$B \cup B1 \cup B2:=\{(x,y) \ | (x,y) \in \mathbb{N} \}$}
	 

\end{itemize}


 \end{document}